\documentclass[11pt]{article}
\usepackage{fullpage}
\usepackage{graphics,epsfig,color}
\usepackage{wrapfig}
\usepackage{times}
\usepackage{setspace}
\usepackage{amsmath,amsthm,amssymb}
\usepackage{url}
\usepackage{fancyhdr}
\pagestyle{fancy}


\newtheorem{theorem}{Theorem}[section]
\newtheorem{corollary}{Corollary}[section]
\newtheorem{lemma}{Lemma}[section]
\newtheorem{problem}{Problem}
\newtheorem{definition}{Definition}[section]
\newtheorem{observation}{Observation}[section]
\newtheorem{example}{Example}[section]
\newtheorem{openproblem}{Open Problem}[section]
\newtheorem{fact}{Fact}[section]

\newcommand{\qedsymb}{\hfill{\rule{2mm}{2mm}}}
\newenvironment{proofsketch}
{
	\begin{trivlist}
	\item[\hspace{\labelsep}{\noindent Proof Sketch: }]
}{\qedsymb\end{trivlist}}



%the following few lines until usepackage{algorithm2e} is to avoid the
%conflicts of algorithm2e with other packages.
\makeatletter
\newif\if@restonecol
\makeatother
\let\algorithm\relax
\let\endalgorithm\relax
\usepackage[ruled,vlined,linesnumbered]{algorithm2e}

\newcommand{\remove}[1]{}



%--------------------------------


\begin{document}

	\renewcommand{\headrulewidth}{0.4pt}
	
	\fancyhead[L]{\bf CSCD320 Homework1, Winter 2012, 
	Eastern Washington University. Cheney, Washington. \\
	\bigskip Name: Eric Fode\hspace{40mm}EWU ID:005301214}
	
	
	\noindent
	\rule[0.1cm]{16.5cm}{0.01cm} 
	
	\vspace*{2mm}
	
	

	
	
	\bigskip
	\bigskip
	\noindent{\bf Solution for Problem 1}
	%---------------------------------------
	\begin{proof} Prove that $T(n) = 2T(n/3) +n^2 = \theta(n^2)$ using induction\\
	 	Claim: if $T(n) = 2T(n/3) +n^2$ $T(n) \leq c(n^2)$ for some constant c\\
	 	Assume:	\\
	 	$$T(1) = 2$$
	 	$$c = 2$$
		Base Case of 3:	
		 	\begin{align*}
		 		T(3) &= 2T(3/3) + 3^2\\
		 		&=2 * 2 + 9\\
		 		&= 13
		 	\end{align*}
		Hypothesis: assume $T(m) \leq c*m^2$ is true for all $m = 3,...,n$\\
		Inductive step: when $m = n + 1$
		\begin{align*}
			T(n+1) &= 2T(n/3) + n^2\\
			&=2T(\frac{n+1}{3}) + (n+1)^2\\
			&<= c (n+1)^2 + (n+1)^2\\
			&< (n+1)^2(c+1)
		\end{align*}
	\end{proof}
	\newpage
	\noindent{\bf Solution for Problem 2}
	\begin{enumerate}
	\item $T(n) = 4T(n/2) + 3n^2 - 9n$\\
	$a = 4, b =2, f(n) = 3n^2 -9n$\\
	case 2 is satisfied since
	\begin{align*}
	3n^2 - 9n &= \theta(n^log_{2}4)\\
	&= \theta(n^2 log n)
	\end{align*}
	so
	$$T(n) = \theta(n^2)$$
	\item $T(n) = 4T(n2) + 2n^3 - 100 n^2$\\
	$a = 4, b =2,f(n) = 2n^3 - 100 n^2$\\
	case 3 is satisfied since
	$$2n^3 - 100n^2 = \Omega(n^2+1)$$
	and
	$$4 * ((n) ^ 3 - 50n^2) \leq 1/2 (2n^3 - 100 n^2) $$
	so
	$$T(n) = \theta(n^3)$$
	\item $T(n) = 4T(n/2) + n + 5 log n$\\
	$a = 4 , b= 2, f(n) = n + 5 log n$\\
	case 1 is satisfied since 
	$$n + 5 log n = O(n^{2 -1})$$
	so
	$$T(n) = \theta(n^2)$$
	\item $T(n) = 8T(n/2) + n^2 + n log n$\\
	$a = 8 , b= 2, f(n) = n^2 + n log n$\\
	case 1 is satisfied since
	$$n^2 + n log n = O(n^{3 -1})$$
	so
	$$T(n) = \theta(n^3)$$
	\item $T(n) = 8T(n/2) + 4n^3 + 5n^2$\\
	$a = 8 , b= 2, f(n) = 4n^3 + 5n^2$\\
	case 2 is satisfied since
	$$4n^3 + 5n^2 = \theta(n^{3})$$
	so
	$$T(n) = \theta(n^3log n)$$
	
	\item $T(n) = 8T(n/2) + 2^-10 n^4 + 6n^3$\\
	$a = 8 , b= 2, f(n) = 2^-10 n^4 + 6n^3$\\
	case 3 is satisfied since
	$$ 2^-10 n^4 + 6n^3 = \Omega(n^{3}+1)$$
	and
	$$ (frac{2^-10 n}{2})^4 + 3n^3 \leq c (2^-10 n^4 + 6n^3)$$
	when n is large
	so
	$$T(n) = \theta(n^4)$$	
	\end{enumerate}
	%---------------------------------------
	\bigskip
	
	\noindent{\bf Solution for Problem  3}
	
	
	\bigskip
	\noindent{\bf Solution for Problem 4}
	\begin{enumerate}
	\item Idea: navigate to bottom of tree and step back one level
	compair subtree left max, subtree right max and this tree total return max of these three
	repeate until top node is reached
	\item  Pseudocode:
		
	\begin{algorithm}[H]
		\NoCaptionOfAlgo
		\caption{\bf maxtree($tree$)}
		
		\KwIn{A tree}
		\KwResult{The root node of the largest subtree}
		\Begin{
					\If{$tree == null$}
					{
						return 0\;
					}
					\If{$tree.leaf == true$}
					{
						$tree.head.value = tree.head.weight$\;
						return $tree.head$\;
					}
					$leftReturn \longleftarrow maxtree(tree.left)$\;
					$rightReturn \longleftarrow maxtree(tree.right)$\;
					$tree.head.value \longleftarrow tree.left.value + tree.right.value + tree.head.weight$\;
					
					\If{$leftReturn.value > rightReturn.value$ {\bf and} $leftReturn.value > tree.head.value$}
					{
						\Return $leftReturn$\;
					}
					\If{$rightReturn.value > leftReturn.value$ {\bf and} $rightReturn.value > tree.head.value$}
					{
						\Return $rightReturn$\;
					}
					\If{$tree.head.value > leftReturn.value$ {\bf and} $tree.head.value > rightReturn.value$}
					{
						\Return $tree.head$\;
					}
					\If{$leftReturn.value == rightReturn.value$ {\bf and} $rightReturn.value > tree.head.value$}
					{
						\Return $rightReturn$\;
					}
					
				}
		}
	\end{algorithm}
	\item Analyze:
	This algorithm can be represented by the following recurance 
	$$T(n) = T(n/2)+ c 
	%---------------------------------------
	\bigskip
	
	\noindent{\bf Solution for Problem 5}
	
	
	
\end{document}




